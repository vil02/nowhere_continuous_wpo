\documentclass[12pt]{article}
\usepackage[english]{babel}
\usepackage[utf8]{inputenc}
\usepackage{inputenc}
\usepackage[a4paper, left=2.0cm, right=2.0cm, top=2.0cm, bottom=2.0cm, headsep=0.0cm]{geometry}
\usepackage{amssymb}
\usepackage{amsmath}
\usepackage{amsthm}
\usepackage{lscape}

\usepackage[ddmmyyyy,hhmmss]{datetime}
\renewcommand{\dateseparator}{.}

\usepackage{hyperref}

\newcommand{\R}{\mathbb{R}}
\newcommand{\Q}{\mathbb{Q}}
\newcommand{\Z}{\mathbb{Z}}
\newcommand{\N}{\mathbb{N}}

\renewcommand{\leq}{\leqslant}
\renewcommand{\geq}{\geqslant}


\newcommand{\define}[1]{\textit{#1}}
\newcommand{\paren}[1]{\! \left(#1 \right)}
\newcommand{\set}[1]{\! \left\{#1 \right\}}


\theoremstyle{plain}
\newtheorem{theorem}{Theorem}
\newtheorem{lemma}[theorem]{Lemma}
\newtheorem{corollary}[theorem]{Corollary}
\newtheorem{proposition}[theorem]{Proposition}

\theoremstyle{definition}
\newtheorem{definition}[theorem]{Definition}

\theoremstyle{remark}
\newtheorem{remark}[theorem]{Remark}
\newtheorem{example}[theorem]{Example}

\begin{document}
\noindent Piotr Idzik \hfill ver. \today\ \currenttime{}

\vspace*{1cm}
This note contains an example of a nowhere continuous weakly Picard operator.
Such function contradicts Lemma 1 in~\cite{VasileBerinde2011}.

\begin{definition}[{{\cite[Definition 2]{VasileBerinde2011}}}]
    Let $X$ be a metric space.
    We say that a mapping $f \colon X \to X$ is a \define{weakly Picard operator} if
    for each $x \in X$ the sequence $(f^n(x))_{n \in \N}$
    is convergent to $b_x \in X$, which is a fixed point of $f$.
\end{definition}

\begin{lemma}[{{\cite[Lemma 1]{VasileBerinde2011}}}]
    Let $f \colon X \to X$ be a weakly Picard operator and let $b \in X$ be a fixed point of $f$.
    Then $f$ is continuous at point $b$. 
\end{lemma}

\begin{example}
    For $q \in \Q$ let $n_q \in \Z$ and $d_q \in \N$ be such that $q = \frac{n_q}{d_q}$ and $\gcd\paren{n_q, d_q} = 1$.
    Consider a function $f \colon \R \to \R$ defined as
    \begin{equation*}
        f(x) = \left\{ \begin{array}{ll}
            0, & x \in \R \setminus \Q,\\
            d_x, & x \in \Q.
               \end{array} \right.
    \end{equation*} 
    Notice that $f(\R) = \N_0$ and $f(\N_0) = \set{1}$, hence for all $x \in \R$ we get $f\paren{f\paren{x}} = 1 = f\paren{1}$.
    Furthermore $f$ is discontinuous at each point of $\R$.       
\end{example}

\bibliography{bib_data}{}
\bibliographystyle{plain}
\end{document}
 